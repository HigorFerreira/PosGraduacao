% Configurações para o pacote minted
% Deixa o fundo do código levemente cinza, similar a muitos editores
% \setminted{
%     bgcolor=black!5,
%     linenos,
%     xleftmargin=2mm,
%     tabsize=2,
%     breaklines,
%     autogobble,
%     numbersep=5pt
% }

\subsection{Funções de Ordem Superior}
Funções de ordem superior são funções que recebem outras funções como argumento, retornam funções como resultado, ou ambas as coisas. Essa característica é essencial no paradigma funcional, promovendo a composição e reutilização de lógica de forma declarativa e sem efeitos colaterais. Em Python, esse conceito é suportado nativamente por meio de funções como \texttt{map}, \texttt{filter}, \texttt{reduce} e também por funções definidas pelo usuário.

\subsubsection{Definição e Conceito}
Uma função de ordem superior é qualquer função que manipula outras funções como dados. Isso permite encapsular comportamentos, criar pipelines de transformação e escrever código mais expressivo.

\subsubsection*{1. Seleção do Problema}

Deseja-se resolver o seguinte problema computacional com base na lista de trabalho \texttt{lista = [ 5, 12, 14, 0, 1, 2, 24, 49, 40, 3, 7 ]}:
\begin{itemize}
    \item Ordenar a lista de números
    \item Filtrar apenas os números pares da lista ordenada
    \item Calcular o quadrado de cada número filtrado
\end{itemize}

\subsubsection*{2. Implementação Comparada}

\textbf{Abordagem Imperativa}:

\begin{listing}[!ht]
\begin{minted}{python}
lista = [ 5, 12, 14, 0, 1, 2, 24, 49, 40, 3, 7 ]
ordenada = sorted(lista)
resultado = []
for numero in ordenada:
    if numero % 2 == 0:
        resultado.append(numero**2)
\end{minted}
\caption{Solução imperativa}
\label{listing:hof_imperativa}
\end{listing}

\textbf{Abordagem Funcional com Funções de Ordem Superior}:

\begin{listing}[!ht]
\begin{minted}{python}
lista = [ 5, 12, 14, 0, 1, 2, 24, 49, 40, 3, 7 ]
resultado = list(map(lambda x: x**2, filter(lambda x: x % 2 == 0, sorted(lista))))
\end{minted}
\caption{Solução funcional com \texttt{map()}, \texttt{filter()} e \texttt{sorted()}}
\label{listing:hof_funcional}
\end{listing}

\subsubsection*{3. Análise Quantitativa}
A
\begin{itemize}
    \item \textbf{Imperativa}: 4 linhas de código (sem contar a inicialização da lista)
    \item \textbf{Funcional}: 1 linha de código (sem contar a inicialização da lista)
\end{itemize}

\subsubsection*{4. Análise Qualitativa}

% \begin{table}[H]
%     \centering
%     \caption{Comparação qualitativa entre abordagens}
%     \begin{tabular}{|l|c|c|}
%         \hline
%         \textbf{Critério} & \textbf{Imperativa} & \textbf{Funcional} \\
%         \hline
%         Facilidade de compreensão & Alta para iniciantes & Requer familiaridade com HOFs \\
%         \hline
%         Legibilidade & Alta, pois descreve o "como" & Concisa, pois descreve o "quê" \\
%         \hline
%         Manutenibilidade & Modificações são passo a passo & Pode exigir reescrita da expressão \\
%         \hline
%         Clareza estrutural & Lógica explícita e detalhada & Expressão enxuta e aninhada \\
%         \hline
%     \end{tabular}
% \end{table}