% File: main.tex
% Date: 2015, April
% Authors: Gustavo Peixoto, Leon Lima.
% Description: Latex template file for seminar short papers presented at PPG-EM/UERJ.
% Version: 1.0
% Available online on: www.gesar.uerj.br

%% PREAMBLE
\documentclass[date,twocolumn,a4paper]{ppgem} 
\usepackage{amsmath}
\usepackage{enumitem}
\usepackage{pgfplots}
\usepackage{tikz}
\usepackage{pgf-pie}
\usepackage{minted}
\usepackage[numbers, sort&compress]{natbib}  % citações numéricas por padrão
% \usepackage{biblatex}



\setlist{nolistsep}


%% TITLE
\Title{
Python: Paradigma Funcional Aplicado à Transformação de Dados
}

% PLEASE, DO NOT CHANGE THE PAGENUMBER{2}
\PageNumber{2}

% %% INSTITUTION
% \InstA{Rio de Janeiro State University} 
% \InstB{} 

\pagestyle{fancy}


\newenvironment{pythoncode}[1][]{%
    \minted[%
        linenos,
        xleftmargin=2mm,
        tabsize=2,
        breaklines,
        autogobble,
        numbersep=5pt,
        #1 % Allow optional extra parameters
    ]{python}%
}{%
    \endminted%
}

\begin{document}
    \thispagestyle{plain}
    \makeheader
    \renewcommand{\listingscaption}{Código}

    %% KEYWORDS
    % \begin{keywords}
    % Max of 4 keywords, comma separeted, {\LaTeX} typesetting, standardization.
    % \end{keywords}

    %% BODY
    \section*{RESUMO}
    Resumo aqui

    \section{INTRODUÇÃO}


    \section{OBJETIVO}
    O objetivo deste estudo é escrever código manutenível e de fácil compreensão se utilizando da abordagem funcional
    afim de demonstrar os efeitos positivos da mesma.

    % region Metodologia
    \section{METODOLOGIA}
    % O objetivo desta pesquisa é escrever código mantenível e de fácil compreensão, a fim de demonstrar os efeitos positivos do paradigma funcional. Para isso, serão comparadas (através de exemplos práticos) as abordagens de resolução de problemas sob as perspectivas do paradigma não funcional e funcional. Essa comparação permitirá avaliar as diferenças entre os paradigmas e observar características como:

    % * Facilidade de compreensão
    % * Facilidade de manutenção
    % * Clareza do código
    % * Entre outros aspectos relevantes
    Este estudo adotará uma abordagem comparativa entre os paradigmas funcional e imperativo, analisando soluções para problemas comuns em ambas as abordagens. A metodologia consistirá em:
    \begin{enumerate}
        \item Seleção de Problemas: Escolha de um problema representativo, computacionalmente resolvível.
        \item Implementação Comparada: Desenvolvimento das soluções em na abordagem funcional e imperativa
        \item Análise Quantitativa: Avaliação de aspectos como:
        \begin{enumerate}
            \item Quantidade de linhas de código
            % \item Tempo de execução
            % \item Uso de métricas como complexidade ciclomática e linhas de código (LOC) para apoio quantitativo.
        \end{enumerate}
        \item Análise Qualitativa: Avaliação de aspectos como:
        \begin{enumerate}
            \item Facilidade de compreensão
            \item Legibilidade (mediante revisão por pares)
            \item Manutenibilidade (tempo para introduzir modificações)
            \item Clareza (análise de estrutura e redundância)
        \end{enumerate}
        \item Ferramentas: Todo o trabalho será desenvolvido utilizando a linguagem python
            e o ambiente de \textit{notebooks} do \textit{jupyter}\footnote{
                 Ambiente web interativo para criar documentos com código, texto (Markdown), fórmulas e gráficos. Usa formato JSON (.ipynb) organizado em células de entrada/saída\cite{wiki_jupyter_notebook}.
            }
    \end{enumerate}

    % endregion

    % region Fundamentos Teóricos
    \section{FUNDAMENTOS TEÓRICOS}
        O Paradigma de Programação Funcional (abreviado do inglês como \textbf{FP}, \textit{\textbf{Functional Programming}}), é um modo de pensar
        a resolução dos problemas computáveis a partir da composição de funções\cite{queiroz_func_prog}.\\
        Linguagens que oferecem total suporte a este paradigma geralmente permitem que funcções sejam atribuídas a variáveis, passadas como parâmetros
        e retornadas de outras funções.
    \subsection{Funções puras}
        Pode-se dizer que uma função é pura quando retorna sempre a mesma saída, dados os mesmos argumentos\cite{queiroz_func_prog}. Em resumo, as funções
        puras não podem alterar nenhum estado externo ao seu escopo\footnote{Contexto onde variáveis/expressões são acessíveis. Escopos filhos herdam dos pais, mas não o contrário. Funções criam escopos isolados (ex.: variáveis internas não são acessíveis externamente)\cite{mdn_escopo}.}.\\
        \textbf{Exemplo:}
        Digamos que queira-se implementar um contador para uso geral conforme código abaixo. Os incrementos de valores sempre são dados através da função
        \textbf{incrementa} definida na linha 3.
        \begin{listing}[!ht]
            \begin{minted}[linenos,xleftmargin=2mm,tabsize=2,breaklines,autogobble,numbersep=5pt]{python}
                contador = 0

                def incrementa(valor):
                    contador += valor
                    return contador

                print(incrementa(5))
                # Saída: 5 (esperado)
                print(incrementa(2))
                # Saída: 8 (esperado)
            \end{minted}
            \caption{Exemplo de função impura}
            \label{listing:impura}
        \end{listing}
        \\Observa-se nas linhas 7 e 9 que as chamadas à função \textbf{incrementa} retorna os resultados esperados.\\
        Agora, digamos que em algum ponto do programa a variável \textbf{contador} seja alterada conforme o
        Código \ref{listing:impura-retorno}:

        \begin{listing}[!ht]
            \begin{minted}[linenos,xleftmargin=2mm,tabsize=2,breaklines,autogobble,numbersep=5pt,firstnumber=12]{python}
                contador = 100
                print(incrementa(5))
                # Saída: 105 (inesperado)
            \end{minted}
            \caption{Retorno inesperado de função impura}
            \label{listing:impura-retorno}
        \end{listing}\pagebreak

        Persebe-se na saída da linha 13, que a chamada da função \textbf{incrementa} com parâmetro 5 retorna resultados imprevisíveis. Ao comparar
        a linha 13 com a linha 7 do Código \ref{listing:impura}, nota-se claramente a quebra do conceito de função pura. Isto introduz comportamentos inconssistentes
        que podem levar a \textit{bugs}\footnote{
            Erro em sistemas eletrônicos/software que causa comportamentos inesperados (ex.: travamentos, resultados incorretos). Pode surgir no código-fonte, frameworks, SO ou compiladores. Comum em computação, jogos e cibernética\cite{wiki_bug}.
        } inesperados no código. Bugs desta natureza geralmente são difíceis de rastrear, a situação é ainda
        mais sensível em códigos de natureza \textit{\textbf{Multi-Threading}}\footnote{
            Paradigma que aumenta a eficiência do sistema ao executar múltiplas threads/tarefas simultaneamente, assim como o multiprocessamento. Essencial para a computação moderna\cite{wiki_multithreading} (adaptado).
        }.\\
        Em resumo: Qualquer função que altere estados fora de seu escopo é impura por natureza, pois a alteração de estados globais geram escopos compartilhados em que um mesmo recurso pode ser concorrentemente disputado.
        É este tipo de abordagem que leva a problemas de deadlock\footnote{
            Impasse em que processos ficam bloqueados mutuamente, cada um esperando por um recurso retido por outro. Comum em SOs e bancos de dados, ocorre mesmo com recursos não-preemptíveis (ex.: dispositivos, memória), independente da quantidade disponível. Pode envolver threads em um único processo\cite{wiki_deadlock}.
        } e os mais derivados problemas em computação concorrente e parelela, tornando o uso de semáforos, filas e etc obrigatórios, adicionando grandes complexidades ao código fonte.

    \subsubsection{Purificando a função}
    O Código \ref{listing:purificada} demonstra a versão purificada da função \textbf{incrementa} na linha 3:
    \begin{listing}[!ht]
        \begin{minted}[linenos,xleftmargin=2mm,tabsize=2,breaklines,autogobble,numbersep=5pt]{python}
            contador = 0

            def incrementa(original, valor):
                return original + valor


            print(contador:=incrementa(contador, 5))
            # Saída: 5 (esperado)
            print(contador:=incrementa(contador, 2))
            # Saída: 8 (esperado)

            contador = 100
            print(contador:=incrementa(contador, 5))
            # Saída: 105 (esperado)
        \end{minted}
        \caption{Função purificada}
        \label{listing:purificada}
    \end{listing}\linebreak

    Nota-se que as saídas nas linhas 7, 9 e 13 são as mesmas da versão anterior do código. A diferença é que agora a função
    \textbf{incrementa} não altera nenhuma variável/estado externo ao seu escopo. Sempre que \textbf{incrementa} é chamada,
    a variável \textbf{contador} é atualizada com o novo valor do retorno da função. Observa-se agora que na linha 13 a saída
    105 é esperada, pois o parâmetro não depente mais somente do valor 5 a ser incrementado, mas também do valor original de
    \textbf{contador}.
    Desta forma, a função torna-se pura por não alterar escopos externos a ela. De certa forma, pode-se dizer que a função está a trabalhar
    com o conceito de \textbf{imutabilidade} dentro dos limites de seu escopo (embora \textbf{contador} seja reatribuído diversas vezes).

    \subsection{Imutabilidade}
        A imutabilidade se refere à impossibilidade de mudança de estado de um objeto/varíavel no programa. Em outras palavras: "Dizemos que uma variável é imutável, se após um valor ser vinculado a essa variável, a linguagem não permitir que lhe seja associado outro valor." (\citeauthor{queiroz_func_prog}, \citeyear{queiroz_func_prog})
        Imutabilidade (em se tratando de um objeto), é um objeto cujo estado não pode ser modificado após sua criação. (Wikipédia, 2025)\cite{wiki_immutable}
    
    % endregion
    
    % region Prática
    \section{PRÁTICA}

    \subsection{List Comprehensions}
    As compreensões de lista (\textit{list comprehensions}) são uma forma prática e expressiva de criar listas em Python a partir de sequências iteráveis. Com uma única linha de código, é possível aplicar transformações e filtros aos elementos, de maneira mais clara e concisa do que com laços tradicionais. Elas se alinham aos princípios do paradigma funcional por evitarem efeitos colaterais, privilegiarem a imutabilidade e expressarem a transformação de dados de forma declarativa.

    \subsubsection{Estrutura Sintática}
    A forma geral de uma compreensão de lista é:

    \begin{listing}[!ht]
        \begin{minted}[linenos,xleftmargin=2mm,tabsize=2,breaklines,autogobble,numbersep=5pt]{python}
            [expressao for item in iteravel if condicao]
        \end{minted}
        \caption{Forma geral de uma list comprehension}
        \label{listing:2}
    \end{listing}

    Cada parte da expressão representa:
    \begin{itemize}
        \item \textbf{expressao}: a transformação aplicada a cada item
        \item \textbf{item in iteravel}: a iteração sobre os dados
        \item \textbf{if condicao}: (opcional) aplica um filtro
    \end{itemize}

    \subsubsection{Comparativo com o Paradigma Imperativo}
    \textbf{Imperativo}:

    \begin{listing}[!ht]
        \begin{minted}[linenos,xleftmargin=2mm,tabsize=2,breaklines,autogobble,numbersep=5pt]{python}
            resultado = []
            for x in range(10):
                if x % 2 == 0:
                    resultado.append(x * x)
        \end{minted}
        \caption{List comprehension - Sequencial/imperativa}
        \label{listing:2}
    \end{listing}
    \textbf{Funcional com list comprehension}:
    \begin{listing}[!ht]
        \begin{minted}[linenos,xleftmargin=2mm,tabsize=2,breaklines,autogobble,numbersep=5pt]{python}
            resultado = [x * x for x in range(10) if x % 2 == 0]
        \end{minted}
        \caption{List comprehension - Forma funcional}
        \label{listing:2}
    \end{listing}

    A versão funcional torna o código mais claro e expressivo.
    \subsubsection{Integração com Funções de Alta Ordem\footnote{Veja o capítulo 5.2}}
    List comprehensions podem substituir combinações de \texttt{map()} e \texttt{filter()}, mantendo a clareza:

    \begin{listing}[!ht]
        \begin{minted}[linenos,xleftmargin=2mm,tabsize=2,breaklines,autogobble,numbersep=5pt]{python}
            # map + filter
            list(map(lambda x: x*x, filter(lambda x: x % 2 == 0, range(10))))

            # list comprehension
            [x*x for x in range(10) if x % 2 == 0]
        \end{minted}
        \caption{List comprehension - Comparação com Funções de Alta Ordem}
        \label{listing:2}
    \end{listing}

    \subsubsection{Casos Avançados e Boas Práticas}
    \begin{itemize}
    \item \textbf{Aninhamento}:
    \begin{minted}[linenos,xleftmargin=2mm,tabsize=2,breaklines,autogobble,numbersep=5pt]{python}
    [[i*j for j in range(1, 4)] for i in range(1, 4)]
    \end{minted}
    \item \textbf{Condicional ternário}:
    \begin{minted}[linenos,xleftmargin=2mm,tabsize=2,breaklines,autogobble,numbersep=5pt]{python}
    ['par' if x % 2 == 0 else 'impar' for x in range(5)]
    \end{minted}
    \item \textbf{Múltiplos \texttt{for}}:
    \begin{minted}[linenos,xleftmargin=2mm,tabsize=2,breaklines,autogobble,numbersep=5pt]{python}
    [(x, y) for x in [1, 2] for y in [10, 20]]
    \end{minted}
    \end{itemize}

    \subsubsection{Aplicação em Transformação de Dados}
    Um exemplo clássico na análise de dados:

    \begin{listing}[!ht]
        \begin{minted}[linenos,xleftmargin=2mm,tabsize=2,breaklines,autogobble,numbersep=5pt]{python}
            import csv

            with open("dados.csv") as f:
                reader = csv.DictReader(f)
                dados = [int(row["idade"]) for row in reader if row["ativo"] == "sim"]
        \end{minted}
        \caption{List comprehension - Aplicação em Transformação de Dados}
        \label{listing:2}
    \end{listing}

    Esse exemplo mostra uma transformação funcional e imutável sobre os dados lidos de um CSV.

    \subsubsection{Considerações Finais}
    As compreensões de lista são bastante usadas em Python por facilitarem a escrita de um código mais claro e direto. Elas contribuem para um estilo mais funcional e ajudam a manter o código legível, conciso e fácil de testar. Por isso, são uma boa escolha em tarefas de transformação de dados, especialmente quando queremos montar pipelines simples e objetivos. Quando usadas com bom senso, tornam o código mais fácil de entender e manter no dia a dia.

    \subsection{Funções de ordem superior}
    (Feliper)
    \subsection{Composição de Funções}
    (Rodolfo)
    \subsection{Pipelines de Transformação}
    % endregion

    % \cite{amarante2001}
    \section{REFERÊNCIAS}\\
    \bibliographystyle{plainnat}                % ou abbrvnat, unsrtnat, etc.
    \bibliography{MinhasReferencias}

\end{document}
