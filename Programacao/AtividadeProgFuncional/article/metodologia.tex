
\section{METODOLOGIA}
% O objetivo desta pesquisa é escrever código mantenível e de fácil compreensão, a fim de demonstrar os efeitos positivos do paradigma funcional. Para isso, serão comparadas (através de exemplos práticos) as abordagens de resolução de problemas sob as perspectivas do paradigma não funcional e funcional. Essa comparação permitirá avaliar as diferenças entre os paradigmas e observar características como:

% * Facilidade de compreensão
% * Facilidade de manutenção
% * Clareza do código
% * Entre outros aspectos relevantes
Este estudo adotará uma abordagem comparativa entre os paradigmas funcional e imperativo, analisando soluções para problemas comuns em ambas as abordagens. A metodologia consistirá em:
\begin{enumerate}
    \item Seleção de Problemas: Escolha de um problema representativo, computacionalmente resolvível.
    \item Implementação Comparada: Desenvolvimento das soluções em na abordagem funcional e imperativa
    \item Análise Quantitativa: Avaliação de aspectos como:
    \begin{enumerate}
        \item Quantidade de linhas de código
        % \item Tempo de execução
        % \item Uso de métricas como complexidade ciclomática e linhas de código (LOC) para apoio quantitativo.
    \end{enumerate}
    \item Análise Qualitativa: Avaliação de aspectos como:
    \begin{enumerate}
        \item Facilidade de compreensão
        \item Legibilidade (mediante revisão por pares)
        \item Manutenibilidade (tempo para introduzir modificações)
        \item Clareza (análise de estrutura e redundância)
    \end{enumerate}
    \item Ferramentas: Todo o trabalho será desenvolvido utilizando a linguagem python
        e o ambiente de \textit{notebooks} do \textit{jupyter}\footnote{
             Ambiente web interativo para criar documentos com código, texto (Markdown), fórmulas e gráficos. Usa formato JSON (.ipynb) organizado em células de entrada/saída\cite{wiki_jupyter_notebook}.
        }
\end{enumerate}