\subsection{List Comprehensions}
As compreensões de lista (\textit{list comprehensions}) são uma forma prática e expressiva de criar listas em Python a partir de sequências iteráveis. Com uma única linha de código, é possível aplicar transformações e filtros aos elementos, de maneira mais clara e concisa do que com laços tradicionais. Elas se alinham aos princípios do paradigma funcional por evitarem efeitos colaterais, privilegiarem a imutabilidade e expressarem a transformação de dados de forma declarativa.

\subsubsection{Estrutura Sintática}
A forma geral de uma compreensão de lista é:

\begin{listing}[!ht]
    \begin{minted}[linenos,xleftmargin=2mm,tabsize=2,breaklines,autogobble,numbersep=5pt]{python}
        [expressao for item in iteravel if condicao]
    \end{minted}
    \caption{Forma geral de uma list comprehension}
    \label{listing:2}
\end{listing}

Cada parte da expressão representa:
\begin{itemize}
    \item \textbf{expressao}: a transformação aplicada a cada item
    \item \textbf{item in iteravel}: a iteração sobre os dados
    \item \textbf{if condicao}: (opcional) aplica um filtro
\end{itemize}

\subsubsection{Comparativo com o Paradigma Imperativo}
\textbf{Imperativo}:

\begin{listing}[!ht]
    \begin{minted}[linenos,xleftmargin=2mm,tabsize=2,breaklines,autogobble,numbersep=5pt]{python}
        resultado = []
        for x in range(10):
            if x % 2 == 0:
                resultado.append(x * x)
    \end{minted}
    \caption{List comprehension - Sequencial/imperativa}
    \label{listing:2}
\end{listing}
\textbf{Funcional com list comprehension}:
\begin{listing}[!ht]
    \begin{minted}[linenos,xleftmargin=2mm,tabsize=2,breaklines,autogobble,numbersep=5pt]{python}
        resultado = [x * x for x in range(10) if x % 2 == 0]
    \end{minted}
    \caption{List comprehension - Forma funcional}
    \label{listing:2}
\end{listing}

A versão funcional torna o código mais claro e expressivo.
\subsubsection{Integração com Funções de Alta Ordem\footnote{Veja o capítulo 5.2}}
List comprehensions podem substituir combinações de \texttt{map()} e \texttt{filter()}, mantendo a clareza:

\begin{listing}[!ht]
    \begin{minted}[linenos,xleftmargin=2mm,tabsize=2,breaklines,autogobble,numbersep=5pt]{python}
        # map + filter
        list(map(lambda x: x*x, filter(lambda x: x % 2 == 0, range(10))))

        # list comprehension
        [x*x for x in range(10) if x % 2 == 0]
    \end{minted}
    \caption{List comprehension - Comparação com Funções de Alta Ordem}
    \label{listing:2}
\end{listing}

\subsubsection{Casos Avançados e Boas Práticas}
\begin{itemize}
\item \textbf{Aninhamento}:
\begin{minted}[linenos,xleftmargin=2mm,tabsize=2,breaklines,autogobble,numbersep=5pt]{python}
[[i*j for j in range(1, 4)] for i in range(1, 4)]
\end{minted}
\item \textbf{Condicional ternário}:
\begin{minted}[linenos,xleftmargin=2mm,tabsize=2,breaklines,autogobble,numbersep=5pt]{python}
['par' if x % 2 == 0 else 'impar' for x in range(5)]
\end{minted}
\item \textbf{Múltiplos \texttt{for}}:
\begin{minted}[linenos,xleftmargin=2mm,tabsize=2,breaklines,autogobble,numbersep=5pt]{python}
[(x, y) for x in [1, 2] for y in [10, 20]]
\end{minted}
\end{itemize}

\subsubsection{Aplicação em Transformação de Dados}
Um exemplo clássico na análise de dados:

\begin{listing}[!ht]
    \begin{minted}[linenos,xleftmargin=2mm,tabsize=2,breaklines,autogobble,numbersep=5pt]{python}
        import csv

        with open("dados.csv") as f:
            reader = csv.DictReader(f)
            dados = [int(row["idade"]) for row in reader if row["ativo"] == "sim"]
    \end{minted}
    \caption{List comprehension - Aplicação em Transformação de Dados}
    \label{listing:2}
\end{listing}

Esse exemplo mostra uma transformação funcional e imutável sobre os dados lidos de um CSV.

\subsubsection{Considerações Finais}
As compreensões de lista são bastante usadas em Python por facilitarem a escrita de um código mais claro e direto. Elas contribuem para um estilo mais funcional e ajudam a manter o código legível, conciso e fácil de testar. Por isso, são uma boa escolha em tarefas de transformação de dados, especialmente quando queremos montar pipelines simples e objetivos. Quando usadas com bom senso, tornam o código mais fácil de entender e manter no dia a dia.