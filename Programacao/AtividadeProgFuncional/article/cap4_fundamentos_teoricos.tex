\section{FUNDAMENTOS TEÓRICOS}
O Paradigma de Programação Funcional (abreviado do inglês como \textbf{FP}, \textit{\textbf{Functional Programming}}), é um modo de pensar
a resolução dos problemas computáveis a partir da composição de funções\cite{queiroz_func_prog}.\\
Linguagens que oferecem total suporte a este paradigma geralmente permitem que funcções sejam atribuídas a variáveis, passadas como parâmetros
e retornadas de outras funções.
\subsection{Funções puras}
Pode-se dizer que uma função é pura quando retorna sempre a mesma saída, dados os mesmos argumentos\cite{queiroz_func_prog}. Em resumo, as funções
puras não podem alterar nenhum estado externo ao seu escopo\footnote{Contexto onde variáveis/expressões são acessíveis. Escopos filhos herdam dos pais, mas não o contrário. Funções criam escopos isolados (ex.: variáveis internas não são acessíveis externamente)\cite{mdn_escopo}.}.\\
\textbf{Exemplo:}
Digamos que queira-se implementar um contador para uso geral conforme código abaixo. Os incrementos de valores sempre são dados através da função
\textbf{incrementa} definida na linha 3.
\begin{listing}[!ht]
    \begin{minted}[linenos,xleftmargin=2mm,tabsize=2,breaklines,autogobble,numbersep=5pt]{python}
        contador = 0

        def incrementa(valor):
            contador += valor
            return contador

        print(incrementa(5))
        # Saída: 5 (esperado)
        print(incrementa(2))
        # Saída: 8 (esperado)
    \end{minted}
    \caption{Exemplo de função impura}
    \label{listing:impura}
\end{listing}
\\Observa-se nas linhas 7 e 9 que as chamadas à função \textbf{incrementa} retorna os resultados esperados.\\
Agora, digamos que em algum ponto do programa a variável \textbf{contador} seja alterada conforme o
Código \ref{listing:impura-retorno}:

\begin{listing}[!ht]
    \begin{minted}[linenos,xleftmargin=2mm,tabsize=2,breaklines,autogobble,numbersep=5pt,firstnumber=12]{python}
        contador = 100
        print(incrementa(5))
        # Saída: 105 (inesperado)
    \end{minted}
    \caption{Retorno inesperado de função impura}
    \label{listing:impura-retorno}
\end{listing}\pagebreak

Persebe-se na saída da linha 13, que a chamada da função \textbf{incrementa} com parâmetro 5 retorna resultados imprevisíveis. Ao comparar
a linha 13 com a linha 7 do Código \ref{listing:impura}, nota-se claramente a quebra do conceito de função pura. Isto introduz comportamentos inconssistentes
que podem levar a \textit{bugs}\footnote{
    Erro em sistemas eletrônicos/software que causa comportamentos inesperados (ex.: travamentos, resultados incorretos). Pode surgir no código-fonte, frameworks, SO ou compiladores. Comum em computação, jogos e cibernética\cite{wiki_bug}.
} inesperados no código. Bugs desta natureza geralmente são difíceis de rastrear, a situação é ainda
mais sensível em códigos de natureza \textit{\textbf{Multi-Threading}}\footnote{
    Paradigma que aumenta a eficiência do sistema ao executar múltiplas threads/tarefas simultaneamente, assim como o multiprocessamento. Essencial para a computação moderna\cite{wiki_multithreading} (adaptado).
}.\\
Em resumo: Qualquer função que altere estados fora de seu escopo é impura por natureza, pois a alteração de estados globais geram escopos compartilhados em que um mesmo recurso pode ser concorrentemente disputado.
É este tipo de abordagem que leva a problemas de deadlock\footnote{
    Impasse em que processos ficam bloqueados mutuamente, cada um esperando por um recurso retido por outro. Comum em SOs e bancos de dados, ocorre mesmo com recursos não-preemptíveis (ex.: dispositivos, memória), independente da quantidade disponível. Pode envolver threads em um único processo\cite{wiki_deadlock}.
} e os mais derivados problemas em computação concorrente e parelela. Desta forma o uso de semáforos, filas e etc acabam sendo obrigatórios, adicionando grandes complexidades ao código fonte.

\subsubsection{Purificando a função}
O Código \ref{listing:purificada} demonstra a versão purificada da função \textbf{incrementa} na linha 3:
\begin{listing}[!ht]
\begin{minted}[linenos,xleftmargin=2mm,tabsize=2,breaklines,autogobble,numbersep=5pt]{python}
    contador = 0

    def incrementa(original, valor):
        return original + valor


    print(contador:=incrementa(contador, 5))
    # Saída: 5 (esperado)
    print(contador:=incrementa(contador, 2))
    # Saída: 8 (esperado)

    contador = 100
    print(contador:=incrementa(contador, 5))
    # Saída: 105 (esperado)
\end{minted}
\caption{Função purificada}
\label{listing:purificada}
\end{listing}\linebreak

Nota-se que as saídas nas linhas 7, 9 e 13 são as mesmas da versão anterior do código. A diferença é que agora a função
\textbf{incrementa} não altera nenhuma variável/estado externo ao seu escopo. Sempre que \textbf{incrementa} é chamada,
a variável \textbf{contador} é atualizada com o novo valor do retorno da função. Observa-se agora que na linha 13 a saída
105 é esperada, pois o parâmetro não depente mais somente do valor 5 a ser incrementado, mas também do valor original de
\textbf{contador}.
Desta forma, a função torna-se pura por não alterar escopos externos a ela. De certa forma, pode-se dizer que a função está a trabalhar
com o conceito de \textbf{imutabilidade} dentro dos limites de seu escopo (embora \textbf{contador} seja reatribuído diversas vezes).

\subsection{Imutabilidade}
A imutabilidade se refere à impossibilidade de mudança de estado de um objeto/varíavel no programa. Em outras palavras: "Dizemos que uma variável é imutável, se após um valor ser vinculado a essa variável, a linguagem não permitir que lhe seja associado outro valor." (\citeauthor{queiroz_func_prog}, \citeyear{queiroz_func_prog}, p.25).\linebreak
Segundo \citeauthor{queiroz_func_prog} (\citeyear{queiroz_func_prog}) a imutabilidade tem grande importância para que máquinas que se comunicam entre
si em um ambiente de programação distribuída não tenham que lidar com estados inconsistentes.\linebreak
Ainda, segundo \citeauthor{goncalves_func_prog} (\citeyear{goncalves_func_prog}) não é necessário sincronizar o acesso ao código quando a imutabilidade está aplicada, pois leituras concorrentes são inofensivas, tornando este cenário ideal para o uso de multi-threading sem que hajam os efeitos adversos
do acesso sumltâneo a recursos compaartilhados (mutáveis).