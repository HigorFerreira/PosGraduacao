% File: main.tex
% Date: 2015, April
% Authors: Gustavo Peixoto, Leon Lima.
% Description: Latex template file for seminar short papers presented at PPG-EM/UERJ.
% Version: 1.0
% Available online on: www.gesar.uerj.br

%% PREAMBLE
\documentclass[date,twocolumn,a4paper]{ppgem} 
\usepackage{amsmath}
\usepackage{enumitem}
\usepackage{pgfplots}
\usepackage{tikz}
\usepackage{pgf-pie}
\usepackage{minted}
% \usepackage{biblatex}



\setlist{nolistsep}


%% TITLE
\Title{
Python: Paradigma Funcional Aplicado à Transformação de Dados
}

% PLEASE, DO NOT CHANGE THE PAGENUMBER{2}
\PageNumber{2}

% %% INSTITUTION
% \InstA{Rio de Janeiro State University} 
% \InstB{} 

\pagestyle{fancy}


\newenvironment{pythoncode}[1][]{%
    \minted[%
        linenos,
        xleftmargin=2mm,
        tabsize=2,
        breaklines,
        autogobble,
        numbersep=5pt,
        #1 % Allow optional extra parameters
    ]{python}%
}{%
    \endminted%
}

\begin{document}
    \thispagestyle{plain}
    \makeheader



    %% KEYWORDS
    % \begin{keywords}
    % Max of 4 keywords, comma separeted, {\LaTeX} typesetting, standardization.
    % \end{keywords}

    %% BODY
    \section*{RESUMO}
    Resumo aqui

    \section{INTRODUÇÃO}


    \section{OBJETIVO}
    O objetivo deste estudo é escrever código manutenível e de fácil compreensão se utilizando da abordagem funcional
    afim de demonstrar os efeitos positivos da mesma.

    \section{FUNDAMENTOS TEÓRICOS}
    De acordo com W3Schools\cite{w3schools_python}, ``Python é uma linguagem de programação popular. Python pode ser usado no servidor para criar aplicações web.''
    \subsection{Imutabilidade}
    Imutabilidade (em se tratando de um objeto), é um objeto cujo estado não pode ser modificado após sua criação. (Wikipédia, 2025)\cite{wiki_immutable}
    \subsection{Funções puras}

    \section{METODOLOGIA}
    % O objetivo desta pesquisa é escrever código mantenível e de fácil compreensão, a fim de demonstrar os efeitos positivos do paradigma funcional. Para isso, serão comparadas (através de exemplos práticos) as abordagens de resolução de problemas sob as perspectivas do paradigma não funcional e funcional. Essa comparação permitirá avaliar as diferenças entre os paradigmas e observar características como:

    % * Facilidade de compreensão
    % * Facilidade de manutenção
    % * Clareza do código
    % * Entre outros aspectos relevantes
    Este estudo adotará uma abordagem comparativa entre os paradigmas funcional e imperativo, analisando soluções para problemas comuns em ambas as abordagens. A metodologia consistirá em:
    \begin{enumerate}
        \item Seleção de Problemas: Escolha de um problema representativo, computacionalmente resolvível.
        \item Implementação Comparada: Desenvolvimento das soluções em na abordagem funcional e imperativa
        \item Análise Quantitativa: Avaliação de aspectos como:
        \begin{enumerate}
            \item Quantidade de linhas de código
            \item Tempo de execução
            % \item Uso de métricas como complexidade ciclomática e linhas de código (LOC) para apoio quantitativo.
        \end{enumerate}
        \item Análise Qualitativa: Avaliação de aspectos como:
        \begin{enumerate}
            \item Facilidade de compreensão
            \item Legibilidade (mediante revisão por pares)
            \item Manutenibilidade (tempo para introduzir modificações)
            \item Clareza (análise de estrutura e redundância)
        \end{enumerate}
        \item Ferramentas: Todo o trabalho será desenvolvido utilizando a linguagem python
            e o ambiente de \textit{notebooks} do \textit{jupyter}\footnote{O que é notebook do jupyter?}
    \end{enumerate}




    \renewcommand{\listingscaption}{Código}

    \section{PRÁTICA}

    Testando a escrita e apresentação do código python no artigo:

    \begin{listing}[!ht]
        \begin{minted}[linenos,xleftmargin=2mm,tabsize=2,breaklines,autogobble,numbersep=5pt]{python}
            palavras_vogais = { palavra: sum([1 for letra in palavra if letra in "aeiouáéíóúãõâêô"]) for palavra in lista_de_palavras }
        \end{minted}
        \caption{Exemplo de contagem de vogais em palavras (funcional)}
        \label{listing:2}
    \end{listing}

    (Israel)
    \subsection{List Comprehensions}
    As compreensões de lista (\textit{list comprehensions}) são uma forma prática e expressiva de criar listas em Python a partir de sequências iteráveis. Com uma única linha de código, é possível aplicar transformações e filtros aos elementos, de maneira mais clara e concisa do que com laços tradicionais. Elas se alinham aos princípios do paradigma funcional por evitarem efeitos colaterais, privilegiarem a imutabilidade e expressarem a transformação de dados de forma declarativa.

    \subsubsection{Estrutura Sintática}
    A forma geral de uma compreensão de lista é:

    \begin{listing}[!ht]
        \begin{minted}[linenos,xleftmargin=2mm,tabsize=2,breaklines,autogobble,numbersep=5pt]{python}
            [expressao for item in iteravel if condicao]
        \end{minted}
        \caption{Forma geral de uma list comprehension}
        \label{listing:2}
    \end{listing}

    Cada parte da expressão representa:
    \begin{itemize}
        \item \textbf{expressao}: a transformação aplicada a cada item
        \item \textbf{item in iteravel}: a iteração sobre os dados
        \item \textbf{if condicao}: (opcional) aplica um filtro
    \end{itemize}

    \subsubsection{Comparativo com o Paradigma Imperativo}
    \textbf{Imperativo}:

    \begin{listing}[!ht]
        \begin{minted}[linenos,xleftmargin=2mm,tabsize=2,breaklines,autogobble,numbersep=5pt]{python}
            resultado = []
            for x in range(10):
                if x % 2 == 0:
                    resultado.append(x * x)
        \end{minted}
        \caption{List comprehension - Sequencial/imperativa}
        \label{listing:2}
    \end{listing}
    \textbf{Funcional com list comprehension}:
    \begin{listing}[!ht]
        \begin{minted}[linenos,xleftmargin=2mm,tabsize=2,breaklines,autogobble,numbersep=5pt]{python}
            resultado = [x * x for x in range(10) if x % 2 == 0]
        \end{minted}
        \caption{List comprehension - Forma funcional}
        \label{listing:2}
    \end{listing}

    A versão funcional torna o código mais claro e expressivo.
    \subsubsection{Integração com Funções de Alta Ordem\footnote{Veja o capítulo 5.2}}
    List comprehensions podem substituir combinações de \texttt{map()} e \texttt{filter()}, mantendo a clareza:

    \begin{listing}[!ht]
        \begin{minted}[linenos,xleftmargin=2mm,tabsize=2,breaklines,autogobble,numbersep=5pt]{python}
            # map + filter
            list(map(lambda x: x*x, filter(lambda x: x % 2 == 0, range(10))))

            # list comprehension
            [x*x for x in range(10) if x % 2 == 0]
        \end{minted}
        \caption{List comprehension - Comparação com Funções de Alta Ordem}
        \label{listing:2}
    \end{listing}

    \subsubsection{Casos Avançados e Boas Práticas}
    \begin{itemize}
    \item \textbf{Aninhamento}:
    \begin{minted}[linenos,xleftmargin=2mm,tabsize=2,breaklines,autogobble,numbersep=5pt]{python}
    [[i*j for j in range(1, 4)] for i in range(1, 4)]
    \end{minted}
    \item \textbf{Condicional ternário}:
    \begin{minted}[linenos,xleftmargin=2mm,tabsize=2,breaklines,autogobble,numbersep=5pt]{python}
    ['par' if x % 2 == 0 else 'impar' for x in range(5)]
    \end{minted}
    \item \textbf{Múltiplos \texttt{for}}:
    \begin{minted}[linenos,xleftmargin=2mm,tabsize=2,breaklines,autogobble,numbersep=5pt]{python}
    [(x, y) for x in [1, 2] for y in [10, 20]]
    \end{minted}
    \end{itemize}

    \subsubsection{Aplicação em Transformação de Dados}
    Um exemplo clássico na análise de dados:

    \begin{listing}[!ht]
        \begin{minted}[linenos,xleftmargin=2mm,tabsize=2,breaklines,autogobble,numbersep=5pt]{python}
            import csv

            with open("dados.csv") as f:
                reader = csv.DictReader(f)
                dados = [int(row["idade"]) for row in reader if row["ativo"] == "sim"]
        \end{minted}
        \caption{List comprehension - Aplicação em Transformação de Dados}
        \label{listing:2}
    \end{listing}

    Esse exemplo mostra uma transformação funcional e imutável sobre os dados lidos de um CSV.

    \subsubsection{Considerações Finais}
    As compreensões de lista são bastante usadas em Python por facilitarem a escrita de um código mais claro e direto. Elas contribuem para um estilo mais funcional e ajudam a manter o código legível, conciso e fácil de testar. Por isso, são uma boa escolha em tarefas de transformação de dados, especialmente quando queremos montar pipelines simples e objetivos. Quando usadas com bom senso, tornam o código mais fácil de entender e manter no dia a dia.

    \subsection{Funções de ordem superior}
    (Feliper)
    \subsection{Composição de Funções}
    (Rodolfo)
    \subsection{Pipelines de Transformação}


    \cite{amarante2001}
    \section{REFERÊNCIAS}
    \bibliographystyle{plain}
    \bibliography{MinhasReferencias}

\end{document}
