\section{CASO DE USO: ÁRVORE BINÁRIA DE PESQUISA}

O Código \ref{listing:abp} abaixo demonstra a instanciação de uma árvore binária de pesquisa, os detalhes da implementação
serão discutidos adiante. O objetivo em um primeiro momento é demonstrar a facilidade de usar implementações funcionais
sobre a árvore afim de realizar algumas operações mais comuns.

\begin{listing}[!ht]
    \begin{minted}[linenos,xleftmargin=2mm,tabsize=2,breaklines,autogobble,numbersep=5pt]{python}
    from Btree import BinaryTree, Node
    from toolz import compose, compose_left
    
    tree = BinaryTree([ 5, 2, 6, 10, 1, 3, 4, 9 ])
    \end{minted}
    \caption{Árvore Binária de Pesquisa}
    \label{listing:abp}
\end{listing}

Nota-se na linha 4 que a árvore \textbf{tree} é criada a partir de uma lista de inteiros. O objetivo
é analisar as chamadas e usos de funções sobre operações comuns desta estrutura de dados, tais como:

\begin{enumerate}
    \item Percorrimento da árvore em ordem
    \item Uso da Busca em Profundidade, do inglês: \textit{Depth First Search} (DFS)\footnote{
        Algoritmo de exploração de grafos que percorre ramificações até o fim antes de retroceder (backtracking). Usado em árvores, grafos e labirintos – estudado desde o século XIX por Trémaux (\citeauthor{wiki_dfs}, \citeyear{wiki_dfs})\cite{wiki_dfs}.
    }, para obter informações, como:
    \begin{enumerate}
        \item Checar a presença de um determinado nó na árvore
        \item Listar os nós folha da árvore
        \item Percorrer determinado nó até a raiz
        \item Calcular a altura da árvore
        \item Calcular o balanceamento da árvore
    \end{enumerate}
\end{enumerate}

\subsection{Usos}
\subsubsection{Percorrimento em ordem}
Conforme mostrado no Código \ref{listing:abp}, a árvore binária é incializada com uma lista de inteiros totalmente desordenada.
Um dos usos mais comuns da Árvore Binária são os percorrimentos em:
\begin{itemize}
    \item\textbf{Pré-Ordem}
    \item\textbf{Em-Ordem}
    \item\textbf{Pós-Ordem}
\end{itemize}

O percorrimento \textbf{Em-Ordem} pode ser utilizado para percorrer todos os nós da árvore de forma que os nós mais à esquerda são visitados primeiro. Dada a natureza de inserção dos itens em uma Árvore Binária,
sabe-se que os items cujos valores sejam menores estarão à esquerda, ao passo que os maiores à direita. Desta forma, pode-se imprimir a lista totalmente ordenada conforme mostrado no Côdigo \ref{listing:abp-in-order}:

\begin{listing}[!ht]
    \begin{minted}[linenos,xleftmargin=2mm,tabsize=2,breaklines,autogobble,numbersep=5pt]{python}
    tree.inOrderWalk(lambda node: print(node.value, end=" "))
    # Saída: 1 2 3 4 5 6 9 10 
    \end{minted}
    \caption{Percorrimento em ordem}
    \label{listing:abp-in-order}
\end{listing}

Nota-se no código \ref{listing:abp-in-order} o nível de abstração quando aplicada a programação funcional, pois a medida que o percorrimento ocorre, cada nó encontrado chama a funçåo lambda\footnote{
    Explicar a função lambda
} passada como segundo parâmetro de \textbf{inOrderWalk}. A função então desempenha o papel de imprimir o valor do nó
Observa-se que ao contrário de simplesmente imprimir o nó, poderia fazer-se qualquer outra operação com o mesmo, como empilhá-los para formar uma nova lista
ordenada, por exemplo.

\subsubsection{Depth First Search (DFS)}

\begin{listing}[!ht]
    \begin{minted}[linenos,xleftmargin=2mm,tabsize=2,breaklines,autogobble,numbersep=5pt]{python}
    tree.dfs(lambda node: node.value == 1)
    # Saída: Node(1)
    tree.dfs(lambda node: node.value == 11)
    # Saída: None
    \end{minted}
    \caption{DFS: }
    \label{listing:dfs}
\end{listing}

\subsection{Implementação}
\subsubsection{Node}
\subsubsection{BinaryTree}
