% File: main.tex
% Date: 2015, April
% Authors: Gustavo Peixoto, Leon Lima.
% Description: Latex template file for seminar short papers presented at PPG-EM/UERJ.
% Version: 1.0
% Available online on: www.gesar.uerj.br

%% PREAMBLE
\documentclass[date,twocolumn,a4paper]{ppgem} 
\usepackage{amsmath}
\usepackage{enumitem}
\usepackage{pgfplots}
\usepackage{tikz}
\usepackage{pgf-pie}
\usepackage{minted}



\setlist{nolistsep}


%% TITLE
\Title{
Python: Paradigma Funcional Aplicado à Transformação de Dados
}

% PLEASE, DO NOT CHANGE THE PAGENUMBER{2}
\PageNumber{2}

% %% INSTITUTION
% \InstA{Rio de Janeiro State University} 
% \InstB{} 

\pagestyle{fancy}


\newenvironment{pythoncode}[1][]{%
    \minted[%
        linenos,
        xleftmargin=2mm,
        tabsize=2,
        breaklines,
        autogobble,
        numbersep=5pt,
        #1 % Allow optional extra parameters
    ]{python}%
}{%
    \endminted%
}

\begin{document}
\thispagestyle{plain}
\makeheader



%% KEYWORDS
% \begin{keywords}
% Max of 4 keywords, comma separeted, {\LaTeX} typesetting, standardization.
% \end{keywords}

%% BODY
\section*{RESUMO}
Resumo aqui

\section{INTRODUÇÃO}
\subsection{Imutabilidade}
\subsection{Funções puras}

\section{OBJETIVO}
O objetivo deste estudo é escrever código manutenível e de fácil compreensão se utilizando da abordagem funcional
afim de demonstrar os efeitos positivos da mesma.

\section{METODOLOGIA}
% O objetivo desta pesquisa é escrever código mantenível e de fácil compreensão, a fim de demonstrar os efeitos positivos do paradigma funcional. Para isso, serão comparadas (através de exemplos práticos) as abordagens de resolução de problemas sob as perspectivas do paradigma não funcional e funcional. Essa comparação permitirá avaliar as diferenças entre os paradigmas e observar características como:

% * Facilidade de compreensão
% * Facilidade de manutenção
% * Clareza do código
% * Entre outros aspectos relevantes
Este estudo adotará uma abordagem comparativa entre os paradigmas funcional e imperativo, analisando soluções para problemas comuns em ambas as abordagens. A metodologia consistirá em:
\begin{enumerate}
    \item Seleção de Problemas: Escolha de um problema representativo, computacionalmente resolvível.
    \item Implementação Comparada: Desenvolvimento das soluções em na abordagem funcional e imperativa
    \item Análise Quantitativa: Avaliação de aspectos como:
    \begin{enumerate}
        \item Quantidade de linhas de código
        \item Tempo de execução
        % \item Uso de métricas como complexidade ciclomática e linhas de código (LOC) para apoio quantitativo.
    \end{enumerate}
    \item Análise Qualitativa: Avaliação de aspectos como:
    \begin{enumerate}
        \item Facilidade de compreensão
        \item Legibilidade (mediante revisão por pares)
        \item Manutenibilidade (tempo para introduzir modificações)
        \item Clareza (análise de estrutura e redundância)
    \end{enumerate}
    \item Ferramentas: Todo o trabalho será desenvolvido utilizando a linguagem python
        e o ambiente de \textit{notebooks} do \textit{jupyter}\footnote{O que é notebook do jupyter?}
\end{enumerate}




\section{PRÁTICA}

Testando a escrita e apresentação do código python no artigo:

\renewcommand{\listingscaption}{Código}

\begin{listing}[!ht]
\begin{minted}[linenos,xleftmargin=2mm,tabsize=2,breaklines,autogobble,numbersep=5pt]{python}
lista_de_palavras = [ "Python", "é", "uma", "ótima", "linguagem" ]
for palavra in lista_de_palavras:
vogais = 0
for letra in palavra:
    if letra in "aeiouáéíóúãõâêô":
    vogais += 1
print(f'{palavra} ==> Tem {vogais} vogais')
\end{minted}
\caption{Exemplo de contagem de vogais em palavras}
\label{listing:2}
\end{listing}

\pagebreak
Curiosidade: Este código pode ser reescrito em apenas duas linhas utilizando a abordagem funcional.

% \mint{python}|palavras_vogais = { palavra: sum([1 for letra in|\noindent
% \mint{python}|palavra if letra in "aeiouáéíóúãõâêô"]) for|\noindent
% \mint{python}|palavra in lista_de_palavras }|\noindent

\begin{listing}[!ht]
\begin{minted}[linenos,xleftmargin=2mm,tabsize=2,breaklines,autogobble,numbersep=5pt]{python}
palavras_vogais = { palavra: sum([1 for letra in palavra if letra in "aeiouáéíóúãõâêô"]) for palavra in lista_de_palavras
\end{minted}
\caption{Exemplo de contagem de vogais em palavras (funcional)}
\label{listing:2}
\end{listing}


\subsection{List Comprehensions}
\subsection{Funções de ordem superior}
\subsection{Composição de Funções}
\subsection{Pipelines de Transformação}


\end{document}
