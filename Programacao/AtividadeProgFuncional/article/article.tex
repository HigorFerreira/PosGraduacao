% File: main.tex
% Date: 2015, April
% Authors: Gustavo Peixoto, Leon Lima.
% Description: Latex template file for seminar short papers presented at PPG-EM/UERJ.
% Version: 1.0
% Available online on: www.gesar.uerj.br

%% PREAMBLE
\documentclass[date,twocolumn,a4paper]{ppgem} 
\usepackage{amsmath}
\usepackage{enumitem}
\usepackage{pgfplots}
\usepackage{tikz}
\usepackage{pgf-pie}
\usepackage{minted}



\setlist{nolistsep}


%% TITLE
\Title{
Python: Paradigma Funcional Aplicado à Transformação de Dados
}

% PLEASE, DO NOT CHANGE THE PAGENUMBER{2}
\PageNumber{2}

% %% INSTITUTION
% \InstA{Rio de Janeiro State University} 
% \InstB{} 

\pagestyle{fancy}

\begin{document}
\thispagestyle{plain}
\makeheader



%% KEYWORDS
% \begin{keywords}
% Max of 4 keywords, comma separeted, {\LaTeX} typesetting, standardization.
% \end{keywords}

%% BODY
\section*{RESUMO}
Resumo aqui

\section{INTRODUÇÃO}
\subsection{Imutabilidade}
\subsection{Funções puras}

\section{OBJETIVO}
O objetivo deste objeto é escrever código manutenível e de fácil compreensão se utilizando da abordagem funcional
afim de demonstrar os efeitos positivos do mesmo.

\section{METODOLOGIA}
% O objetivo desta pesquisa é escrever código mantenível e de fácil compreensão, a fim de demonstrar os efeitos positivos do paradigma funcional. Para isso, serão comparadas (através de exemplos práticos) as abordagens de resolução de problemas sob as perspectivas do paradigma não funcional e funcional. Essa comparação permitirá avaliar as diferenças entre os paradigmas e observar características como:

% * Facilidade de compreensão
% * Facilidade de manutenção
% * Clareza do código
% * Entre outros aspectos relevantes
Este estudo adotará uma abordagem comparativa entre os paradigmas funcional e imperativo, analisando soluções para problemas comuns em ambas as abordagens. A metodologia consistirá em:
\begin{enumerate}
    \item Seleção de Problemas: Escolha de um problema representativo, computacionalmente resolvível.
    \item Implementação Comparada: Desenvolvimento das soluções em na abordagem funcional e imperativa
    \item Análise Quantitativa: Avaliação de aspectos como:
    \begin{enumerate}
        \item Quantidade de linhas de código
        \item Tempo de execução
        % \item Uso de métricas como complexidade ciclomática e linhas de código (LOC) para apoio quantitativo.
    \end{enumerate}
    \item Análise Qualitativa: Avaliação de aspectos como:
    \begin{enumerate}
        \item Facilidade de compreensão
        \item Legibilidade (mediante revisão por pares)
        \item Manutenibilidade (tempo para introduzir modificações)
        \item Clareza (análise de estrutura e redundância)
    \end{enumerate}
    \item Ferramentas: Todo o trabalho será desenvolvido utilizando a linguagem python
        e o ambiente de \textit{notebooks} do \textit{jupyter}\footnote{O que é notebook do jupyter?}
\end{enumerate}




\section{PRÁTICA}

Testando a escrita e apresentação do código python no artigo:

\renewcommand{\listingscaption}{Código}

\begin{listing}[!ht]
\begin{minted}{python}
lista_de_palavras = [ "Python", "é", "uma",
                    "ótima", "linguagem" ]
                    
for palavra in lista_de_palavras:
vogais = 0
for letra in palavra:
    if letra in "aeiouáéíóúãõâêô":
    vogais += 1
print(f'{palavra} ==> Tem {vogais} vogais')
\end{minted}
\caption{Exemplo de contagem de vogais em palavras}
\label{listing:2}
\end{listing}

\pagebreak
Curiosidade: Este código pode ser reescrito em apenas duas linhas utilizando a abordagem funcional.

% \begin{listing}[!ht]
% \begin{minted}{c}
% #include <stdio.h>
% int main() {
%     printf("Hello, World!"); /*printf() outputs the quoted string*/
%     return 0;
% }
% \end{minted}
% \caption{Hello World in C}
% \label{listing:2}
% \end{listing}

\subsection{List Comprehensions}
\subsection{Funções de ordem superior}
\subsection{Composição de Funções}
\subsection{Pipelines de Transformação}

% region
% \section{DADOS DA PESQUISA}

% Na tabela 1 podemos ver os dados colhidos da pesquisa

% \begin{table}[h!]
%   \centering
%   \caption{Tabela com os dados dividos por gênero}
%   \label{tbl:something}
%   \renewcommand{\arraystretch}{1.5}
%   \begin{tabular}{c c c c}
%     \hline
%     TOTAL & 208.901 & 316.546 & 525.447 \\
%     \hline
%     55 a 59 & 47.264 & 78.516 & 125.780 \\
%     60 a 64 & 46.523 & 70.388 & 116.911 \\
%     65 a 69 & 41.550 & 59.444 & 100.994 \\
%     70 a 74 & 33.293 & 46.776 & 80.069 \\
%     75 a 79 & 22.609 & 32.651 & 55.260 \\
%     80 e + & 17.662 & 28.771 & 46.433 \\    
%     \hline
%   \end{tabular}
% \end{table}

% % =====================================================================

% \section{GRÁFICOS}


% \begin{tikzpicture}
% \begin{axis}[
%     ybar,
%     symbolic x coords={55-59, 60-64, 65-69, 70-74, 75-79, 80+},
%     xtick=data,
%     bar width=8pt,
%     x tick label style={rotate=45, anchor=east},
%     legend style={at={(0.5,-0.15)},
%     anchor=north,legend columns=-1},
%     ymin=0,ymax=50,
%     enlarge x limits=0.15
% ]
% \addplot[draw=blue,fill=blue] coordinates {
%     (55-59, 47.264) (60-64, 46.523) (65-69, 41.55) (70-74, 33.293) (75-79, 22.609) (80+, 17.662)
% };
% \addplot[draw=red,fill=red] coordinates {
%     (55-59, 47.264) (60-64, 46.523) (65-69, 41.55) (70-74, 33.293) (75-79, 22.609) (80+, 17.662)
% };
% \legend{Masculino, Feminino}
% \end{axis}
% \end{tikzpicture}


% \begin{figure}[h!]
%     \centering
%     \caption{Distribuição entre os gêneros}
%     \label{img:pizza}
%     \begin{tikzpicture}
%         \pie[rotate=90, text=legend, radius=2, before number=\phantom, after number=\%, explode=0.1, color={blue!50, red!50}]{
%             39.76/Masculine,
%             60.24/Feminino
%         }
%     \end{tikzpicture}
% \end{figure}
% endregion

\end{document}
