\section{CASO DE USO: ÁRVORE BINÁRIA DE PESQUISA}

O Código \ref{listing:abp} abaixo demonstra a instanciação de uma árvore binária de pesquisa, os detalhes da implementação
serão discutidos adiante. O objetivo em um primeiro momento é demonstrar a facilidade de usar implementações funcionais
sobre a árvore afim de realizar algumas operações mais comuns.

\begin{listing}[!ht]
    \begin{minted}[linenos,xleftmargin=2mm,tabsize=2,breaklines,autogobble,numbersep=5pt]{python}
    from Btree import BinaryTree, Node
    from toolz import compose, compose_left
    
    tree = BinaryTree([ 5, 2, 6, 10, 1, 3, 4, 9 ])
    \end{minted}
    \caption{Árvore Binária de Pesquisa}
    \label{listing:abp}
\end{listing}

Nota-se na linha 4 que a árvire \textbf{tree} é criada a partir de uma lista de inteiros. O objetivo
é analisar as chamadas e usos de funções sobre operações comuns desta estrutura de dados, tais como:

\begin{itemize}
    \item Percorrimento da árvore em ordem
    \item Uso da Busca em Profundidade, do inglês: \textit{Depth First Search} (DFS)\footnote{
        Algoritmo de exploração de grafos que percorre ramificações até o fim antes de retroceder (backtracking). Usado em árvores, grafos e labirintos – estudado desde o século XIX por Trémaux (\citeauthor{wiki_dfs}, \citeyear{wiki_dfs})\cite{wiki_dfs}.
    }, para obter informações, como:
    \begin{itemize}
        \item Checar a presença de um determinado nó na árvore
        \item Listar os nós folha da árvore
        \item Percorrer determinado nó até a raiz
        \item Calcular a altura da árvore
        \item Calcular o balanceamento da árvore
    \end{itemize}
\end{itemize}
