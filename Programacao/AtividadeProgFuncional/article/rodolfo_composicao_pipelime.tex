\subsection{Composição de Funções}
A composição de funções é um conceito no paradigma de programação funcional que permite combinar funções simples para criar outras mais complexas. Essa técnica é essencial para a construção de programas modulares, reutilizáveis e fáceis de testar.

\subsubsection{Definição e Conceito}
A composição de funções é como montar um quebra-cabeça, onde cada peça (função) se encaixa na próxima para formar uma agregação maior para apresentação do resultado. Em termos matemáticos, a composição de funções $f$ e $g$ resulta em uma nova função $h(x) = g(f(x))$, onde a saída de $f$ se torna a entrada de $g$.

\subsubsection{Exemplo Prático: Transformação de Dados}
Deseja-se resolver o seguinte problema computacional com base na lista de trabalho:
\begin{minted}[frame=none, xleftmargin=5mm]{python}
lista = [5, 12, 14, 0, 1, 2, 24, 49, 40, 3, 7]
\end{minted}
As etapas da transformação são:
\begin{itemize}
    \item Ordenar a lista de números.
    \item Filtrar apenas os números pares da lista ordenada.
    \item Calcular o quadrado de cada número filtrado.
\end{itemize}

\subsubsection{Comparativo: Imperativo vs. Funcional}
\textbf{Abordagem Imperativa}:

\begin{codelisting}{Transformação de dados de forma imperativa}
    \label{listing:imperativa}
    \begin{minted}{python}
# Funcao Ordena Lista
def ordenaLista(lista):
  return sorted(lista)

# Funcao Filtra Lista
def filtraLista(lista):
  resultado_filtra = []
  for numero in lista:
    if numero % 2 == 0:
      resultado_filtra.append(numero)
  return resultado_filtra

# Funcao Calcula Lista
def calculaLista(lista):
  resultado_calcula = []
  for numero in lista:
    resultado_calcula.append(numero**2)
  return resultado_calcula

# Estrutura Principal
resultado_lista_ordenada = ordenaLista(lista)
resultado_lista_filtrada = filtraLista(resultado_lista_ordenada)
resultado_lista_calculada = calculaLista(resultado_lista_filtrada)
    \end{minted}
\end{codelisting}

\textbf{Abordagem Funcional com Composição}:
\begin{codelisting}{Transformação de dados com composição funcional}
    \label{listing:funcional}
    \begin{minted}{python}
# Abordagem 1: Composição explícita (aninhada)
resultado_aninhado = calculaLista(filtraLista(ordenaLista(lista)))

# Abordagem 2: Pipeline funcional com map e filter
resultado_pipeline = list(map(lambda x: x**2, 
    filter(lambda x: x % 2 == 0, sorted(lista))))
    \end{minted}
\end{codelisting}

A versão funcional expressa o fluxo de dados de forma mais direta e declarativa.

\subsubsection{Análise dos Resultados}
\textbf{Quantitativa}
\begin{itemize}
    \item \textbf{Imperativa:} 10 linhas de código (sem contar a inicialização, apenas as 3 funções de apoio).
    \item \textbf{Funcional:} 1 linha de código (sem contar a inicialização, para a versão com \texttt{map/filter}).
\end{itemize}

\textbf{Qualitativa}
\begin{itemize}
    \item \textbf{Reutilização:} Funções individuais (\texttt{ordena}, \texttt{filtra}, \texttt{calcula}) podem ser reutilizadas em diferentes composições, aumentando a modularidade.
    \item \textbf{Testabilidade:} Funções puras, sem efeitos colaterais, são fáceis de testar individualmente.
    \item \textbf{Clareza:} A composição de funções pode tornar o código mais legível, pois cada função tem uma responsabilidade específica e o fluxo de dados é explícito.
    \item \textbf{Manutenção:} Funções pequenas e bem definidas são mais fáceis de manter e modificar.
\end{itemize}

\subsubsection{Considerações Finais}
A programação funcional facilita a programação modular, já que permite construir funções complexas a partir de funções simples e programas a partir da composição de outros programas. Uma função é uma regra de correspondência, como uma função matemática que mapeia membros de um conjunto domínio para um conjunto imagem (PELLEGRINI, 2013).

Uma linguagem funcional oferece: 1) um conjunto de funções primitivas; 2) um conjunto de formas funcionais para construir funções complexas; 3) uma operação de aplicação das funções; e 4) estruturas para representar dados (PELLEGRINI, 2013).

Assim, a composição de funções é uma técnica poderosa na programação funcional, permitindo a construção de código modular, reutilizável e fácil de entender e manter, seguindo os princípios de funções puras e imutabilidade.

% \begin{thebibliography}{9}
%     \bibitem{pellegrini2013} PELLEGRINI, T. A. \textit{Paradigmas de programação: uma introdução}. 2013. Disponível em: \url{https://cepein.femanet.com.br/BDigital/arqPics/1911420024P1028.pdf}
%     \bibitem{synapse2021} \textit{Paradigmas de Programação: Uma Introdução}. Editora Synapse, 2021. Disponível em: \url{https://www.editorasynapse.org/wp-content/uploads/2021/03/paradigmas_programacao_uma_introducao_V0.pdf}
% \end{thebibliography}



\subsection{Pipelines de Transformação}
Pipelines de transformação de dados são sequências de etapas que automatizam o processo de preparação e transformação de dados, desde a coleta até a análise. Eles ajudam a garantir que os dados sejam processados de maneira consistente e eficiente.

Imagine uma empresa que coleta dados de vendas de várias fontes: planilhas, sistemas de gerenciamento de clientes e redes sociais. Com o crescente volume dessas informações, o tratamento dos dados torna-se exaustivo, gastando horas limpando, organizando e analisando esses dados manualmente. Muitas vezes, erros passam despercebidos, resultando em decisões baseadas em informações imprecisas.

É nesse cenário caótico que os pipelines de transformação de dados se tornam essenciais. Eles automatizam todo o processo, desde a coleta até a análise, garantindo que os dados sejam consistentes, confiáveis e prontos para serem usados na tomada de decisões estratégicas. Com um pipeline eficiente, as empresas conseguem transformar dados brutos em insights valiosos em questão de minutos, não horas.

\subsubsection{Definição e Conceito}
No paradigma funcional, pipelines de transformação referem-se à aplicação sequencial de funções sobre dados, onde a saída de uma função se torna a entrada da próxima. Essa abordagem, conhecida como composição funcional, permite criar transformações complexas de forma modular e reutilizável, evitando efeitos colaterais e tornando o código mais fácil de entender e manter, especialmente em pipelines de dados ETL (Extração, Transformação, Carregamento).

\subsubsection{Por que usar Pipelines?}
\begin{itemize}
    \item \textbf{Automatização:} Pipelines permitem que o processo de transformação de dados seja automatizado, reduzindo a necessidade de intervenção manual e minimizando erros.
    \item \textbf{Reprodutibilidade:} Uma vez configurado, o pipeline pode ser executado repetidamente com diferentes conjuntos de dados, garantindo que o processo seja consistente.
    \item \textbf{Eficiência:} Pipelines ajudam a economizar tempo, permitindo que as equipes se concentrem na análise e interpretação dos dados, em vez de se perderem em tarefas manuais.
    \item \textbf{Escalabilidade:} Com o aumento do volume de dados, pipelines podem ser escalados para lidar com grandes quantidades de informações sem comprometer a performance.
\end{itemize}

\subsubsection{Características das Pipelines Funcionais}
No contexto de Programação Funcional, os pipelines de transformação apresentam as seguintes características:
\begin{itemize}
    \item \textbf{Funções Puras:} As funções devem ser puras, ou seja, para a mesma entrada, a saída é sempre a mesma, e não há efeitos colaterais (como modificar variáveis externas).
    \item \textbf{Funções como blocos:} Em vez de manipular dados diretamente com laços e condicionais, o paradigma funcional usa funções como blocos de construção. Cada função realiza uma transformação específica.
    \item \textbf{Composição de Funções:} As funções são "encadeadas" de forma que a saída de uma alimenta a entrada da próxima. Isso cria um fluxo de dados através de uma série de transformações.
    \item \textbf{Imutabilidade:} Dados em um pipeline funcional são tipicamente imutáveis. Cada função cria uma nova versão modificada dos dados, preservando a integridade dos dados originais.
    \item \textbf{Paralelização:} A imutabilidade e a natureza pura das funções facilitam a paralelização das operações, permitindo que diferentes etapas sejam executadas simultaneamente.
    \item \textbf{Padrões de projeto:} Funções de ordem superior, como \texttt{map}, \texttt{filter} e \texttt{reduce}, são frequentemente utilizadas para manipular coleções de dados.
\end{itemize}

\subsubsection{Aplicação Prática}
\textbf{Exemplo 1: Pipeline Genérico}

Neste exemplo, dado uma lista de números, implementa-se um pipeline para percorrer a lista e aplicar, para cada elemento, as seguintes funções:
\begin{enumerate}
    \item Adicionar 1 (\texttt{adicionar\_um(x)}).
    \item Multiplicar por 2 (\texttt{multiplicar\_por\_dois(x)}).
\end{enumerate}

\begin{codelisting}{Implementação de um pipeline de funções}
    \label{listing:pipeline_generico}
    \begin{minted}{python}
# Funcao para incrementar um valor "X"
def adicionar_um(x):
  return x + 1

# Funcao para multiplicar um valor "X" por 2
def multiplicar_por_dois(x):
  return x * 2

# Funcao Pipeline que aplica uma lista de funcoes a uma lista de dados
def pipeline(dados, funcoes):
  resultado = []
  for item in dados:
    processado = item
    for funcao in funcoes:
      processado = funcao(processado)
    resultado.append(processado)
  return resultado

# Dados iniciais
numeros = [1, 2, 3, 4, 5]
# Criando o pipeline com a sequencia de transformacoes
transformacoes = [adicionar_um, multiplicar_por_dois]

# Aplicando o pipeline
resultado_final = pipeline(numeros, transformacoes)

# Resultado: [4, 6, 8, 10, 12]
print(resultado_final)
    \end{minted}
\end{codelisting}

\textbf{Exemplo 2: Pipeline com Funções Lambda e Filtro}

Neste exemplo, um pipeline de transformação e filtragem é aplicado a uma lista de entrada para:
\begin{enumerate}
    \item Aplicar a função \texttt{dobro}.
    \item Aplicar a função \texttt{somar\_dez}.
    \item Filtrar os resultados com a função \texttt{filtro} (manter apenas valores $> 10$).
\end{enumerate}

\begin{codelisting}{Pipeline com lambda e filter}
    \label{listing:pipeline_lambda}
    \begin{minted}{python}
# Funcao de pipeline definida no exemplo anterior
def pipeline(dados, funcoes):
    # ... (mesma implementacao)
    resultado = []
    for item in dados:
        processado = item
        for funcao in funcoes:
            processado = funcao(processado)
        resultado.append(processado)
    return resultado

entrada = [1, 2, 3, 4, 5]
dobro = lambda x: x * 2
somar_dez = lambda x: x + 10
filtro = lambda x: x > 10

# Criando o pipeline de transformacao
transformacoes = [dobro, somar_dez]
resultado_transformado = pipeline(entrada, transformacoes)

# Aplicando o Filtro apos a transformacao
resultado_filtrado = filter(filtro, resultado_transformado)

# Resultado Final: [12, 14, 16, 18, 20]
print(list(resultado_filtrado))
    \end{minted}
\end{codelisting}

\subsubsection{Benefícios dos Pipelines Funcionais}
\begin{itemize}
    \item \textbf{Reutilização:} Funções podem ser combinadas de diferentes maneiras para criar outras transformações em diferentes pipelines.
    \item \textbf{Testabilidade:} Funções individuais e puras são mais fáceis de testar, pois o resultado é sempre previsível.
    \item \textbf{Clareza:} O fluxo de dados é mais fácil de seguir, pois a lógica é dividida em funções pequenas e independentes.
    \item \textbf{Manutenção:} Modificações em uma parte do pipeline geralmente não afetam outras partes e são mais fáceis de entender.
    \item \textbf{Escalabilidade:} A imutabilidade e a natureza pura facilitam a paralelização e a escalabilidade do pipeline.
\end{itemize}

\subsubsection{Aplicações de Pipelines}
\begin{itemize}
    \item \textbf{ETL:} Pipelines são amplamente usados em sistemas ETL (Extração, Transformação e Carga) para processar grandes volumes de dados.
    \item \textbf{Processamento de imagens:} Podem ser usados para aplicar uma série de transformações em imagens, como redimensionamento, correção de cores e filtros.
    \item \textbf{Análise de dados:} São essenciais para o pré-processamento dos dados antes da análise, como limpeza, agregação e normalização.
\end{itemize}

\subsubsection*{Considerações Finais}
Pipelines de transformação no paradigma de programação funcional oferecem uma abordagem modular, reutilizável e escalável para processar dados, com benefícios significativos em termos de testabilidade, manutenção e desempenho.

% \begin{thebibliography}{9}
%     \bibitem{medium_funcional}
%     Gomes, M. \textit{Programação Funcional: Teoria e Conceitos}. Medium, 2020. Disponível em: \url{https://medium.com/@marcelomg21/programa%C3%A7%C3%A3o-funcional-teoria-e-conceitos-975375cfb010}

%     \bibitem{dio_pipelines}
%     \textit{Descomplicando a Análise de Dados: Conheça os Pipelines}. DIO.me. Disponível em: \url{https://www.dio.me/articles/descomplicando-a-analise-da-dados-conheca-os-pipelines}
    
%     \bibitem{ufc_repo}
%     Queiroz, P. C. G. \textit{Título do Trabalho}. Repositório UFC, 2024. Disponível em: \url{https://repositorio.ufc.br/bitstream/riufc/78403/3/2024_tcc_pcgqueiroz.pdf}

% \end{thebibliography}